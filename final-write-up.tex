\documentclass{article}
\usepackage[top=.5in, bottom=.5in, left=.9in, right=.9in]{geometry}
\usepackage[latin1]{inputenc}
\usepackage{enumerate}
\usepackage{hyperref}
\usepackage{graphics}
\usepackage{graphicx}
\usepackage{caption}
\usepackage{subcaption}
\usepackage{tabularx}
\usepackage{amsmath}
\usepackage{amssymb}
\usepackage{siunitx}
\usepackage{mathtools}
\usepackage{bbm}



\usepackage[authoryear,round]{natbib}

% Use instead of natbib
%\usepackage[backend=bibtex,citestyle=authoryear-comp,natbib=true,sorting=none,hyperref=true,maxnames=2,arxiv=pdf]{biblatex}
%\renewbibmacro{in:}{}
%\addbibresource{/Users/Evan/GitProjects/tex-docs/references.bib}




\newcommand{\obar}[1]{\ensuremath{\overline{ #1 }}}
\newcommand{\iid}{\ensuremath{\stackrel{\textrm{iid}}{\sim}}}
\newcommand{\op}[2]{{\ensuremath{\underset{ #2 }{\operatorname{ #1 }}~}}}
\newcommand{\norm}[1]{{ \ensuremath{ \left\lVert  #1 \right\rVert  }  }}
\newcommand{\cov}{ \ensuremath{ \textrm{cov} } }
\newcommand{\var}{ \ensuremath{ \textrm{var} } }
\newcommand{\tr}{ \ensuremath{ \textrm{trace} } }
\newcommand{\df}{ \ensuremath{ \textrm{df} } }
\newcommand{\R}{ \ensuremath{ \mathbb{R} }}
\newcommand{\indicator}[1]{ \ensuremath{ \mathbbm{1}\left\{ #1 \right\} }   }

\usepackage{xcolor}
\definecolor{darkgreen}{rgb}{0,0.25,0}
\newcommand{\soln}{{\color{red}\textbf{Solution:~}\color{black}}}


\usepackage[formats]{listings}
\lstdefineformat{R}{~={\( \sim \)}}
\lstset{% general command to set parameter(s)
basicstyle=\small\ttfamily, % print whole listing small
keywordstyle=\bfseries\rmfamily,
keepspaces=true,
% underlined bold black keywords
commentstyle=\color{darkgreen}, % white comments
stringstyle=\ttfamily, % typewriter type for strings
showstringspaces=false,
numbers=left, numberstyle=\tiny, stepnumber=1, numbersep=5pt, %
frame=shadowbox,
rulesepcolor=\color{black},
,columns=fullflexible,format=R
} %
\renewcommand{\ttdefault}{cmtt}
% enumerate is numbered \begin{enumerate}[(I)] is cap roman in parens
% itemize is bulleted \begin{itemize}
% subfigures:
% \begin{subfigure}[b]{0.5\textwidth} \includegraphics{asdf.jpg} \caption{} \label{subfig:asdf} \end{subfigure}
\hypersetup{colorlinks=true, urlcolor=blue, linkcolor=blue, citecolor=blue}


\graphicspath{ {C:/Users/Evan/Desktop/} }
\title{\vspace{-6ex}SDS 385: Final Project\vspace{-2ex}}
\author{Evan Ott \\ UT EID: eao466\vspace{-2ex}}
%\date{DATE}
\setcounter{secnumdepth}{0}
\usepackage[parfill]{parskip}



\begin{document}
\maketitle


\tableofcontents


\section{Methods}
We tried two general methods for trying to identify markers for periodontal disease: logistic regression with a
LASSO penalty and the off-the-shelf \texttt{DESeq2} package in R. Within each framework, we also tried
a master data set where each gene is treated separately, along with a ``grouped'' data set, where similar
genes were counted together into biologically-sensible groups.

\subsection{Data}

The master data can be thought of as a matrix $X$ with dimension $N\times P$ where $N$ samples were taken
of $P$ genes from across the microbiome. $X_{i,j}$, determined using RNA-seq, represents the count of how many
times gene $j$ was expressed for sample $i$. Along with this information, we also know $Y_i$, whether sample
$i$ is in the control or treatment group. (Note: We also know which patient the sample comes from, but in
our analysis, we disregarded this and treated all samples as independent -- a simplifying assumption that we
take with a grain of salt)

The grouped data $\tilde{X}$ is then an $N\times \tilde{P}$ matrix, where $\tilde{P} < P$ is the number of groups
of genes. $Y$ remains unchanged.

Here, we have $N=27$ samples, $P=152448$ observed genes, and $\tilde{P}=2227$ groups of genes.

\subsection{Logistic Regression}

For this problem, perhaps without much surprise to those with topic familiarity, the logistic regression model
failed pretty extravagantly. Using a pre-processing step common in using this manner of count data, we applied
the following log transformation:
$$Z_{i,j}=\left\{\begin{array}{cc}
0 & X_{i,j}=0\\
\log_2 (X_{i,j}+0.1) & X_{i,j} > 0
\end{array}\right.$$
Essentially, keep the zeros, but scale the rest of the data on a log scale (adding a small constant to ensure
a count of one is not represented as a zero.

We used stochastic gradient descent to solve this objective:
\begin{align*}
\op{minimize}{\alpha \in \R,~\beta \in \R^p}& \sum_i \norm{y_i - \hat{y}_{\alpha,\beta}(Z_{i,\cdot})}_2^2 + \lambda \norm{\beta}_1\\
\hat{y}_{\alpha, \beta}(x)=&\frac{1}{1+\exp\left(- \alpha - x^\top \beta \right)}
\end{align*}
that is, logistic regression with a LASSO penalty on the non-intercept parameters.

However, on both the ``master'' and ``grouped'' data, we were unable to produce a model that had any semblance
of good predictive power. We used 3-fold cross validation to select the penalty $\lambda$ using the held-out data.
In particular, we considered the mean 0-1 error (number of correct predictions when $\hat{y}$ is rounded to 0 or 1)
and the mean of the absolute error ($\sum_i \left|y_i - \hat{y}_{\alpha,\beta}(Z_{i,\cdot})\right|$). We used 3 folds
simply because we had so few samples $N=27$. Consequently, we used the ``eye test,'' as it were, to identify
reasonable values of $\lambda$ from plotting these cross-validated error estimates. In the grouped case, see Figure \ref{fig:one}

\begin{figure}[h]
\begin{center}
\includegraphics[scale=0.2]{3-fold-CV-kegg.png}
\caption{\label{fig:one}Error using the grouped data versus the LASSO penalty $\lambda$.
In blue, the mean 0-1 error on the held-out set. In red, the mean absolute error on the held-out set.}
\end{center}
\end{figure}

By the ``eye test'' -- we used only 3 folds, so it seems foolish to use the ``most parsimonious within one standard 
error'' trick simply by having a not-believable estimate of the standard error -- the absolute error seems to not 
improve for $\lambda > 10^{-0.4}$ while the 0-1 error does not improve for $\lambda > 10^{+0.6}$. Using the more
restrictive value $\lambda = 10^{+0.6}$, running on the entire grouped data set shows that the predicted values 
$\hat{y}$ are all below 0.5, indicating that the rounded predictions would all be zero! See Figure \ref{fig:two}

\begin{figure}[h]
\begin{center}
\includegraphics[scale=0.15]{kegg-cv-prediction.png}
\caption{\label{fig:two}Predictions for $\lambda=10^{+0.6}$ on the grouped data set versus actual value (0 for control, 1 for treatment).}
\end{center}
\end{figure}

Out of $\tilde{P}=2227$ possible gene groups, this process produced non-zero coefficients for only 4. For the $P=152448$ total genes, the same process (for $\lambda=10^{+0.5}$) produced 25 non-zero coefficients. These
are included in the tables below, but should not be viewed as particularly reliable, for the reasons outlined above.

On the grouped data, we also tried using the raw counts. The cross-validated error plot was certainly more 
interesting, but the results were similar, where the predictions using the entire grouped data set were abysmal.

\subsection{\texttt{DESeq2}}
The R package \texttt{DESeq2} is a standard for analyzing this kind of data. It uses the model below to identify
which genes are more indicative of a control or treatment sample:
\begin{align*}
X_{i,j}\sim& \texttt{Negative-Binomial}(\mu_{i,j}, \alpha_i)\\
\mu_{i,j}=&s_jq_{i,j}\\
\log_2(q_{i,j})=&u_{j,\cdot} \beta_i
\end{align*}
where $u_{j,\cdot}$ comes from the $N\times 3$ design matrix (first column all ones, second column as an indicator
for being in the control group, third column for being in the treatment group), and $\beta_i$ is the quantity of interest,
with $\gamma_i=\beta_{i, 3}-\beta_{i,2}$ being the ``log2 fold change.'' We won't discuss the other parameters
here as they are not of interest in this analysis. For more information, see \S4.1 of \citep{deseq2}.

It's a little unclear exactly how \texttt{DESeq2} handles multiple testing. It says that it uses the Benjamini-Hochberg
procedure to create adjusted $p$-values (such that checking $\textrm{padj}_i<\alpha$ is equivalent to the
Benjamini-Hochberg procedure with FDR controlled at $\alpha$) \citep{benjamini1995controlling}. 
Any gene (gene group) with such an adjusted $p$-value is included in Table \ref{tbl:genes} (Table \ref{tbl:groups}).
However, this 
adjusted $p$-value is not always available -- in cases where a gene/group is filtered by ``independent filtering'' (having a low
mean normalized count),
it is set to \texttt{NA}. See \S1.5.3 of \citep{deseq2} for more.

So, we implemented our own, standard version of the Benjamini-Hochberg procedure
on the non-adjusted $p$-values (that are the result of a Wald test for the log2 fold change). While in principle,
the genes identified in this manner need not have been a subset of those identified in the default setting,
in practice they are. These are seen in the ``Our B-H'' column in the tables.

Finally, we also looked into the ``independent filtering'' which is described in \S3.8 and \S4.7 of \citep{deseq2}.
This essentially tries to determine a cutoff for the mean of the normalized counts where smaller values are
excluded from the analysis. Turning this off (in general) will likely yield fewer rejections once \texttt{DESeq2}'s 
internal Benjamini-Hochberg procedure is applied -- more ``noise'' data will enter in, padding the ``signal'' data,
making it more difficult for a particular signal to make it past the Benjamini-Hochberg procedure (if the significance
level is kept constant). However, if we are happy with fewer rejections (perhaps just a smaller set of genes
to consider during an initial investigation), this becomes useful. The results from turning off the filtering (but using
\texttt{DESeq2}'s specialized Benjamini-Hochberg procedure) are in the ``No Filtering'' column in the tables.


\section{Results}
Tables \ref{tbl:groups} and \ref{tbl:genes} represent the high-level results of this investigation. They indicate
genes/groups of genes identified by logistic regression (that should not be regarded highly for reasons outlined 
above) and by \texttt{DESeq2}. The variants applied to the latter indicate smaller subsets of genes/groups of genes
to consider when trying to investigate any biological basis for periodontal disease. In particular,
dpig\_c\_1\_8  and raer\_c\_9\_1389 seem indicative of positive and negative markers, respectively for disease based
on turning off \texttt{DESeq2}'s independent filtering (Table \ref{tbl:genes}).

\begin{table}[h]
\begin{center}
\begin{tabular}{r|cccl}
Gene group & LR & \texttt{DESeq2} & Our B-H & No Filtering\\
\hline
1.1.1.36 & & - & & \\
1.14.-.-   & & - & &\\
1.14.14.- & & - & & \\
1.14.14.10& & - & -&- \\
1.2.1.39& & - & -& -\\
1.2.99.8& & - & -& -\\
1.3.99.5& & - & -& -\\
1.4.99.-  & Y \\
2.1.1.140& & - &- &- \\
2.7.7.73& & - & & \\
2.7.13.3 & Y \\
3.1.3.21& & - & -& -\\
3.4.11.-& & + & +& +\\
3.4.11.9 & Y \\
3.4.24.-  & Y\\
3.5.1.54& & - & -& -\\
3.6.3.9& & + &+& +\\
4.2.1.153& & - & & \\
5.1.1.7& & - & -& -\\
5.3.3.-& & - &- & -\\
5.4.99.26& & - & -& -\\
6.-.-.-& & - & & \\
6.3.3.3& & - &- &- \\
6.4.1.6& & - & &- \\
\end{tabular}
\caption{\label{tbl:groups}Gene groups identified using the various strategies outlined. A ``Y'' in the first column
indicates that the listed gene group had a non-zero coefficient in the logistic regression method -- this is primarily
recorded for posterity, not because the analysis is trusted. For the remaining columns, a ``-'' indicates that
a high count in that gene group is indicative of being a control sample. A ``+'' indicates that a high count is indicative
of being a treatment sample. A blank indicates that it is not present in that particular model.}
\end{center}
\end{table}

\begin{table}[h]
\begin{center}
\begin{tabular}{r|ccc}
Gene &  \texttt{DESeq2} & B-H & No Filter\\
\hline
abau\_c\_1\_3356 & - &  &   \\
aisr\_c\_20\_1998 & -  &  &   \\
   amas2385\_c\_4\_872 & - &  &   \\
   aot170\_c\_17\_1854 & - &  &   \\
  aot171\_c\_140\_1750 & - &  &   \\
     aot171\_c\_3\_100 & - &  &   \\
   aot172\_c\_99\_1663 & - &  &   \\
   aot448\_c\_6\_964 & +  &  &   \\
  apre\_c\_1\_700 & - &  &   \\
 cdip\_c\_1\_1509 & - &  &   \\
chom\_c\_27\_1386 & - &  &   \\
chom\_c\_55\_2086 & - &  &   \\
  cper\_c\_2\_296   & +  &  &   \\
 cure\_c\_1\_1578 & - &  &   \\
    dpig\_c\_1\_8   & +  &  & +  \\
     efae1080\_c\_1\_913 & - &  &   \\
     esak\_c\_1\_3901 & - &  &   \\
    fnucp\_c\_3\_1442  & + &  &   \\
  fper2555\_c\_2\_652   & +  &  &   \\
   gmor\_c\_19\_1592 & - &  &   \\
 lcat\_c\_18\_1871 & - &  &   \\
    lgas\_c\_1\_25 & - &  &   \\
  lgoo\_c\_14\_911 & - &  &   \\
   lmon\_c\_1\_308 & - &  &   \\
lot107\_c\_9\_1346  &+ &  &   \\ 
  mlot\_c\_1\_5349 & - &  &   \\
  mneo\_c\_1\_1226 & - &  &   \\
  mneo\_c\_1\_1611 & - &  &   \\
  mot186\_c\_3\_1980 & - &  &   \\
      mtub\_c\_1\_788   &+ &  &   \\
      nbac\_c\_3\_267 & - &  &   \\
 NCBIABIX\_c\_1\_1199 & - &  &   \\
opro\_c\_21\_2166 & - &  &   \\
  pend\_c\_1\_176 & - &  &   \\
peno\_c\_75\_2748 & - &  &   \\
pmel\_c\_20\_1104 & - &  &   \\
 pmuls\_c\_6\_923 & - &  &   
     \end{tabular}
  \begin{tabular}{r|ccc}
Gene &  \texttt{DESeq2} &   B-H & No Filter\\
\hline
 pot786\_c\_28\_1621 & - &  &   \\
 pple\_c\_7\_1120 & - &  &   \\
 pstu\_c\_1\_2331 & - &  &   \\
pver\_c\_10\_1377 & - &  &   \\
  raer\_c\_1\_313 & - &  &   \\
raer\_c\_17\_1833 & - &  &   \\
  raer\_c\_2\_422 & - &  &   \\
 raer\_c\_5\_1016 & - &  &   \\
 raer\_c\_9\_1389 & - &  - & -  \\
  rden\_c\_1\_560 & - &  &   \\
    rden1994\_c\_1\_356 & - &  &   \\
    sked\_c\_1\_1556 & - &  &   \\
    sked\_c\_1\_1767 & -  &  &   \\
smal\_c\_1\_1545 & -  &  &   \\
smit\_c\_1\_1759 & -  &  &   \\
smut\_c\_1\_1468 & +  &  &   \\
smut\_c\_1\_1548 & +  &  &   \\
smut\_c\_1\_1794 & + &  &   \\
smut\_c\_1\_371 & +  &  &   \\
smut\_c\_1\_836 & +  &  &   \\
  sot138\_c\_21\_1472 & - &  &   \\
  sot149\_c\_17\_1639 & - &  &   \\
    ssal\_c\_18\_1887 & - &  &   \\
    tmed\_c\_12\_2247 & - &  &   \\
\end{tabular}
\caption{\label{tbl:genes}Gene identified using the various strategies outlined. The interpretation is identical
to that in Table \ref{tbl:groups}. The logistic regression data is not included here, because, like in the grouped data,
it is a) a completely different set of genes and b) has seemingly no predictive power.}
\end{center}
\end{table}



\bibliographystyle{plainnat}
\bibliography{/Users/Evan/GitProjects/tex-docs/references}

% If using biblatex
% \printbibliography

\end{document}